\documentclass[12pt]{article}
\usepackage{lineno}
%\usepackage{ametsoc}
\usepackage{natbib}
\usepackage{a4wide}
\usepackage{amsfonts}
\usepackage{amssymb}

\usepackage{color}
\usepackage[pdftex]{graphicx}
\usepackage[pdftex,colorlinks, citecolor=black,linkcolor=black,urlcolor=black]{hyperref}
\definecolor{darkgreen}{rgb}{0.1,0.4,0.1}
\definecolor{darkblue}{rgb}{0.1,0.1,0.3}



\newif\ifdetail
\global\detailfalse
%\global\detailtrue

\def\bem#1{\colorbox{green}{\em #1}}

\newcommand{\beq}  { \begin{eqnarray} }
\newcommand{\eeq}  { \end{eqnarray}}
\newcommand{\beeq}  { \begin{eqnarray*} }
\newcommand{\eeeq}  { \end{eqnarray*}}
\newcommand{\eq }  [1]{Eq.~(\ref{#1})}  % Eq. (#1)
\newcommand{\eqs }  [1]{Eqs.~(\ref{#1})}
\newcommand{\sect}  [1]{Sec.~(\ref{#1})}  % Eq. (#1)
\newcommand{\fig}  [1]{Fig.~\ref{#1}}   % Fig. #1
\renewcommand{\v}[1]{{\mbox{\boldmath$ #1 $}}}  % Vektor durch Fettdruck
\newcommand{\m}    [1]{{\bf  #1 }}              % Matrix in sans serif
\newcommand{\gz}[1] {{{d #1} \over {dz}}}
\newcommand{\pt}[1]{{\partial_t #1}}
\newcommand{\ptt}[1]{{{\partial^2 #1} \over {\partial t^2}}}
\newcommand{\pttx}[1]{{{\partial^3 #1} \over {\partial t^2 \partial x}}}
\newcommand{\ptty}[1]{{{\partial^3 #1} \over {\partial t^2 \partial y}}}
\newcommand{\pttt}[1]{{{\partial^3 #1} \over {\partial t^3}}}
\newcommand{\ptx}[1]{{{\partial^2 #1} \over {\partial t \partial x}}}
\newcommand{\pty}[1]{{{\partial^2 #1} \over {\partial t \partial y}}}
\newcommand{\pz}[1]{{\partial_z #1}}
\newcommand{\pzz}[1]{{\partial_{zz} #1}}
\newcommand{\px}[1]{{\partial_x #1}}
\newcommand{\py}[1]{{\partial_y #1}}
\newcommand{\pxi}[1]{{\partial_i #1}}
\newcommand{\pxj}[1]{{\partial_j #1}}
\newcommand{\pxk}[1]{{\partial_k #1}}
\newcommand{\pb}[1]{{{\partial #1} \over {\partial b}}}
%\newcommand{\pxy}[1]{{{\partial^2 #1} \over {\partial x \partial y}}}
\newcommand{\pxy}[1]{\partial_{xy} #1}
\newcommand{\pyy}[1]{\partial_{yy} #1}
\newcommand{\pxx}[1]{\partial_{xx} #1}
\newcommand{\kx}{{\bf k} \times}
\newcommand{\dt}[1]{{{d #1} \over {d t}}}
\newcommand{\Dt}[1]{{{D #1} \over {D t}}}
\newcommand{\Div}[1]{ \nabla \cdot #1 }
\renewcommand{\div}[1]{ \nabla_{yz} \cdot #1 }
\newcommand{\Bar}[1]{ \overline{ #1} }
\newcommand{\rvec} [1]{
    \raisebox{-1.5ex}{$\stackrel{\textstyle #1}{\neg}$} }% rotierter Vektor
\newcommand{\nablq}   {\rvec\nabla}                    % rotated nabla operator
\newcommand{\Vec}[1]{\left(\begin{array}{c} #1 \end{array}\right) }
%\renewcommand{\mho}{\mathcal{W}}
\newcommand{\pn}[1]{{{\partial #1} \over {\partial n}}}
\newcommand{\pnn}[1]{{{\partial #1} \over {\partial m}}}
\newcommand{\ps}[1]{{{\partial #1} \over {\partial s}}}
\newcommand{\p}{{\partial}}
\newcommand{\vn}{{\v \nabla}}
\newcommand{\lk}{\left(}
\newcommand{\rk}{\right)}

\newcommand{\tl}{\tilde}


\newcommand{\E}{\mathcal{E}}
\newcommand{\A}{\mathcal{A}}

\newcommand{\ep}{\epsilon}

\begin{document}


\section*{Raytracing model}


This is a numerical model for the propagation and refraction 
of internal gravity waves.
It integrates the six-dimensional energy transfer equation in time. The waves propagate 
under the influence of variable stratification and mean flow.


\subsection*{Energy transfer equation}

The energy transfer equation for internal gravity waves with 
wavenumber vector $\v k$  and
intrinsic frequency  $  \omega_i  $ given by
\beq 
 \omega_i = \sqrt{ (N^2 (\ell^2+k^2) + f^2 m^2)/(k^2+\ell^2+m^2)�}
  ~~,~~
 \v k  = (k,\ell,m)=(k_1,k_2,k_3) 
 \eeq
 which are 
propagating in a variable stratification given by $N(x,y,z)$ and horizontal 
mean flow $\v U(x,y,z)=(U,V,0)$ is given by 
\beq
 \p_t{\E} + \p_{\v x}  (\dot{ \v x}  \, \E) %+ \p_y (\dot y \, \E)+ \p_z (\dot z \, \E) 
 + \p_{\v k}  (\dot{\v  k}  \, \E) % + \p_\ell (\dot \ell \, \E)+ \p_m (\dot m \, \E) 
 =    \omega_i^{-1} \E  \dot \omega_i
\eeq
with group velocities $\dot{ \v x} $ 
 and refraction parameter $\dot{\v  k}$ given by
\beq
\dot{\v x} = \p_{\v k} \omega_e= (  \p_k \omega_e, \p_\ell \omega_e ,  \p_m \omega_e )  
 ~~,~~ \dot{\v  k}  =  - \p_{\v x} \omega_e   = -(  \p_x \omega_e, \p_y \omega_e ,  \p_z \omega_e )  
    \eeq
and extrinsic (Doppler shifted) frequency of encounter
\beq
 \omega_e = \omega_i + \v k \cdot \v U
\eeq
The energy exchange with the mean flow is given by the right hand side of the energy transfer equation
with  
\beq
 \dot \omega_i  &=&    - \v k \cdot  ( \vn_{\v k } \omega_i  \cdot \vn \v U )   -  \dot z  \v k \cdot \p_z \v U
 =- \vn_{\v k }  \omega_i  \cdot  ( \v k  \cdot \vn \v U )  -  \dot z  \v k \cdot \p_z \v U 
\\ &=&  -k_i  (\p_{k_j}  \omega_i)   \p_{x_j} U_i  - \dot z k_i \p_z U_i 
%\\ &=&  -k (\dot x - U )   \p_{x} U   -  l  (\dot x - U )   \p_{x} V  -k  (\dot y -V )   \p_{y} U   - l  (\dot y - V )   \p_{y} V - \dot z k_i \p_z U_i 
%\\ &=&  -k (\dot x - U )   \p_{x} U   -  k  (\dot y - V )   \p_{x} V  - l  (\dot x -U )   \p_{y} U   - l  (\dot y - V )   \p_{y} V - \dot z k_i \p_z U_i 
\\ &=&    -k    \vn_{\v k } \omega_i  \cdot  \vn U  -  \ell    \vn_{\v k } \omega_i  \cdot  \vn   V    - \dot z ( k \p_z U +  \ell \p_Z V) 
\eeq
since $\vn_{\v k }  \omega_i   ||   \v k $.


\subsection*{Numerical grid}

The model is discretised on a staggered C-grid in all six dimensions. 
Energy $E$ is centered in a tracer grid box. On the 
eastern, northern and upper sides of these boxes, the zonal, meridional
and vertical velocities, $\dot x$, $\dot y$, $\dot z$, respectively, are located, and similar for directions $k$, $\ell$, and $m$
and the corresponding velocities.
$N$, $U$, and  $V$ are  given on the same grid position in $\v x$ as $E$ --  i.e. on tracer points.

All indices for the six dimensions are counted positive in the respective direction. This also holds for the vertical coordinate.
Energy is given at points $zt_k$, $k=1,...,K$, where $K$ is the number
of grid points in the vertical. Above $zt_k$ ends the grid cell at the level $zu_k$ which
is given by  $zu_k = (zt_{k+1} + zt_k)/2 $.  The grid box size for the tracer is 
 thus $ \Delta zt_k = zu_k - zu_{k-1}$ and we also define
 $ \Delta zu_k = zt_{k+1} - zt_k$. 
 The vertical velocity $\dot z$ is defined at $zu_k$ and is thus in the center of its box, while the tracer point
 $zt_k$ is displaced relative to the center of its box in case of a non-equidistant grid.
 This is sometimes called a u-centered grid.
 The grid in the other directions  follows equivalent rules and naming convention.
The sea surface is  at $zu_K=0\, \rm m$,
where $K$ is the number of vertical levels.
The number of grid points in $k$, $\ell$, and $m$ direction must be even (or just one),
such that $kt_{K/2} = -kt_{K/2-1}$ represent the smallest wavenumber when $K$ is the number of 
grid points in $k$ direktion, 
and such that $ku_{K/2} = 0$. Corresponding rules apply for $\ell$ and $m$.

\subsection*{Discretisation}

The velocities are calculated from gradients of $\omega_e$, which is given on  the same grid position as $E$.
The velocities $\dot x$ and $\dot k$ are e.g. given by
\beq
  \dot x = \delta^-_{kt} \Bar{  \Bar{\omega_e}^{x+}    } ^{k+} 
  ~,~\dot k = -  \delta^-_{xt} \Bar{  \Bar{\omega_e}^{x+}    } ^{k+} 
\eeq
 with the finite differencing operators
\beq
 \delta^+_{x} h_i = (h_{i+1} - h_{i})/\Delta xt_i
  ~~,~~
 \delta^-_{x} h_i = (h_{i} - h_{i-1})/\Delta xt_i
 \eeq
and 
with the finite averaging operators
\beq
 \Bar{h_{i}}^{x+} = (h_{i} +  h_{i+1} )/2 ~~,~~ 
 \Bar{h_{i}}^{x-} =  (h_{i} +  h_{i-1} ) /2
 \eeq
 and similar for the other dimensions $y$, $z$, $k$, $\ell$, $m$.
This satisfies on the discrete grid the condition $\p_x \dot x + \p_k \dot k = \p_x \p_k \omega_e - \p_k \p_x \omega_e = 0 $
and similar for the other dimensions.

The fluxes in the energy transfer terms are calculated using a second order  advection scheme
with superbee flux limiter.
This scheme introduces a certain/necessary amount of numerical diffusion, 
and ensures positive definite energies. Time stepping is then simply forward in time.

\subsection*{Boundary conditions}

Boundary conditions in $z$ and $m$ are reflection boundary conditions, i.e. 
the upward  energy  flux at the surface $z=0,k=-|k^*| $ is converted to a downward energy flux at  $z=0,k= |k^*| $,
and similar for 
the bottom $z=-h$. 
For all other boundaries energy can flow out if the velocity is directed such,
while there is no inflow of energy.  The energy flux at $k$, $\ell$, and $m$ direction is
integrated and stored in an output file.

\subsection*{Parallelisation}

The model domain is decomposed into sub domains for each processor.
The number of subdomains for each direction in $\v x$ and $\v k$ is specified during run time.

\subsection*{Dimension subsets}

Since integrating the model in  six dimensions is computionally very demanding  it is also possible
to use just one grid point in certain dimensions. E.g.  for the case of horizontal homogenuity, i.e. $\v U = \v U(z)$ 
and $N=N(z)$, the grid points in $x$ and $y$ direction can be set to one.
All wavenumber dimensions can also be set to one, a special model variable for the choice of
wavenumber in this dimensions then needs to be specified to calculate wave frequencies, etc.



\end{document}




